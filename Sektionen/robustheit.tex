\label{robustheit}

Der Begriff Robustheit wird durch das IEEE Standard Glossary of Software Engineering Terminology definiert als \glqq Der Grad, zu dem ein System oder eine Komponente in der Lage ist, unter fehlerhaften Eingaben oder belastenden Umgebungsbedingungen korrekt zu funktionieren\grqq{}(\cite{IEEE.1990}, S. 64).

In zwei unabhängigen Studien wurde der Begriff insofern konkret auf RL übertragen, als dass es darum geht, wie gut RL-Verfahren funktionieren, wenn das Verhalten der Umgebung teilweise unbekannt ist. Das ist insbesondere relevant, weil RL-Modelle zunehmend in der realen physischen Welt angewendet werden, während sie weiterhin aus Kosten- oder Zeitgründen in simulierten Umgebungen trainiert werden. Bei der Anwendung dieser Modelle in der realen Welt kommt es dazu, dass deren Leistungsfähigkeit sinkt, da reale Einsatzszenarien Eigenschaften besitzen, die in der Simulation nicht vollständig abgebildet werden (können). So zum Beispiel in der Navigation basierend auf Kamerabildern, bei der zeitweise das Sichtfeld blockiert sein kann, oder bei der Kollaboration von Robotern und Menschen, bei der die Roboter in der Lage sein müssen, auf die vielschichtigen Absichten des Menschen reagieren zu können\cite{Moos.2022}\cite{Ni.2021}.

Das Ziel beim robusten Reinforcement Learning besteht darin, ein Regelwerk zu finden, das auch unter ungünstigen Bedingungen, möglichst gute Entscheidungen trifft. Aus den Studien geht hervor, dass Robustheit in die Teile des modellierten MDPs aufgeteilt werden kann, das von den Unsicherheiten betroffen ist:

\begin{itemize}
	\item Unsicherheit bezüglich Aktionen: Es wird eine andere Aktion ausgeführt, als die für die sich der Agent entschieden hat.
	\item Unsicherheit bezüglich Beobachtungen: Der Zustand, den der Agent beobachtet, entspricht nicht dem tatsächlichen Zustand der Umgebung.
	\item Unsicherheit bezüglich Dynamik der Umgebung: Die Übergangswahrscheinlichkeiten zwischen den Zuständen sind anders als erwartet.
\end{itemize}

Es existiert eine Reihe von Herangehensweisen, um RL-Verfahren besonders robust zu machen, diese werden im Rahmen dieser Arbeit jedoch nicht weiter betrachtet\cite{Moos.2022}\cite{Ni.2021}.

Da sich die verschiedenen Arten der Unsicherheit auf Aspekte des MDPs beziehen, ergibt sich die Möglichkeit, in dieser Hinsicht nicht nur RL-Verfahren, sondern auch MDP-lösende symbolische Algorithmen auf Robustheit zu untersuchen.
