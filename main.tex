\documentclass[a4paper, 12pt]{scrartcl}
\usepackage{comment}
\usepackage{graphicx}
\usepackage[utf8]{inputenc}
\usepackage[ngerman]{babel}
\usepackage[T1]{fontenc}
\usepackage{fancyhdr}
\usepackage{csquotes}
\pagestyle{fancy}
\usepackage[hidelinks]{hyperref}
\usepackage[center]{caption}
\usepackage[onehalfspacing]{setspace}
\usepackage[backend=bibtex, style=authoryear-ibid]{biblatex}
\usepackage{pgfplots}
\usepackage{listings}
%\usepackage[%
%    style=authoryear,
%    giveninits=true,
%    natbib=true,
%    maxbibnames=99,
%    uniquename=init
%]{biblatex}

\DeclareNameAlias{sortname}{family-given}
\renewcommand*{\multinamedelim}{\addcomma\space}
\renewcommand*{\finalnamedelim}{\addcomma\space}
\renewcommand*{\nameyeardelim}{\addcomma\space}
\setlength{\bibitemsep}{\baselineskip}
\addbibresource{lit.bib}

\begin{document}
    \renewcommand{\figurename}{Abb.}
	\fancyhf{}
	\thispagestyle{empty}
\begin{titlepage}
 \begin{center}
    \includegraphics[width=7.7cm]{logo_h-da_neu} \\ 
  \end{center}

  \begin{center}
    \vspace{0.1cm}
    \huge \textbf{Hochschule Darmstadt}\\
    \vspace{0.4cm}
    \LARGE -- Fachbereich Informatik --
  \end{center}

  \vfill
  \vfill

  \begin{center}
    \LARGE \textbf{Robustheit und Generalisierbarkeit in algorithmischen und Reinforcement Learning gestützten Lösungsansätzen: Eine Fallstudie mit Vier Gewinnt}
  \end{center} 
  
  \vfill
  \vfill

  \begin{center}
	\Large Abschlussarbeit zur Erlangung des akademischen Grades\\
	\vspace{0.3cm}
	\Large Bachelor of Science (B. Sc.)\\
  \end{center}

  \vfill

  \begin{center}
    \Large vorgelegt von\\
    \vspace{0.3cm}
    \Large \textbf{Leo Herrmann}\\
    \vspace{0.3cm}
    \normalsize Matrikelnummer: 1111455
  \end{center}
    \vfill
  \vfill

  \begin{center}
    \begin{tabular}{lll}
      Referentin:    & Prof. Dr. Elke Hergenröther \\
      Korreferent: & Adriatik Gashi				 \\
    \end{tabular}
  \end{center} 
\end{titlepage}

	\newpage
	
	\section{Kurzfassung}
	In dieser Arbeit wird am Beispiel des Brettspiels Vier Gewinnt untersucht, inwiefern symbolische Algorithmen oder Reinforcement-Learning-Verfahren robuster sind. Dazu wurden zwei Agenten implementiert, die das Spiel selbstständig spielen. Beim ersten Agenten kommt der symbolische Alorithmus Monte Carlo Tree Search (MCTS) zum Einsatz. Dem zweiten Agenten liegt das RL-Verfahren Proximal Policy Optimization (PPO) zugrunde. Die Robustheit der Agenten wird quantifiziert, indem der Verlust der Gewinnrate gegen einen zufällig spielenden Agenten gemessen wird, während die zu untersuchenden Agenten zwei verschiedenen Szenarien mit ungünstigen Bedingungen ausgesetzt sind, die auf verschiedene Aspekte von Robustheit abzielen. Im ersten Szenario liegt Unsicherheit bezüglich Aktioinen vor, was bedeutet, dass die Agenten keine vollständige Kontrolle darüber haben, in welche Spalte sie ihre Spielsteine platzieren. Im zweiten Szenario erhalten die Agenten fehlerhafte Informationen über das Spielfeld, somit liegt Unsicherheit bezüglich Beobachtungen vor.

Es wird gezeigt, dass unter den beiden implementierten Agenten der MCTS-Agent im Szenario Unsicherheiten bezüglich Aktionen robuster ist als der PPO-Agent. Für das Szenario Unsicherheiten bezüglich Beobachtungen wurde dies bis zu einem gewissen Ausmaß an Unsicherheiten ebenfalls beobachtet. Die Ergebnisse dieser Arbeit lassen sich jedoch nicht auf den Vergleich von Robustheit zwischen MCTS und PPO unabhängig von der konkreten Implementierung oder dem Anwendungsfall übertragen, da der implementierte PPO-Agent für einen fairen Vergleich unzureichend trainiert wurde. Dafür und um allgemeine Aussagen über den Verlgleich der Robustheit zwischen symbolischen Algorithmen und Reinforcement Leraning zu übertragen, sind weitere Untersuchungen erforderlich.

\newpage
	
	\tableofcontents
	\newpage

    \listoffigures
    \newpage

    \section*{Eigenständigkeitserklärung}
	Ich versichere hiermit, dass ich die vorliegende Arbeit selbständig verfasst und keine anderen als die im Literaturverzeichnis angegebenen Quellen benutzt habe.
Alle Stellen, die wörtlich oder sinngemäß aus veröffentlichten oder noch nicht veröffentlichten Quellen entnommen sind, sind als solche kenntlich gemacht.
Die Zeichnungen oder Abbildungen in dieser Arbeit sind von mir selbst erstellt worden oder mit einem entsprechenden Quellennachweis versehen.


\vspace{4cm}

Darmstadt, 21.03.2023

\hspace*{\fill}\begin{tabular}{@{}l@{}}\hline
\makebox[11cm]{}Leo Herrmann
\end{tabular}
	\newpage
 
	\setcounter{page}{1}
	\fancyfoot[C]{\thepage}
	
	\section{Einleitung}
	Fortschreitende Automatisierung durchdringt zahlreiche Bereiche der Gesellschaft, so zum Beispiel die Fertigungsindustrie, das Gesundheitswesen oder den Straßenverkehr. Zwei fundamentale Ansätze sind dabei regelbasierte Algorithmen und Machine Learning. Die Einsatzbedingungen von Automatisierungssystemen unterscheiden sich häufig von den Bedingungen, unter denen sie entwickelt und getestet werden. Häufig müssen Systeme mit fehlerhaften oder veralteten Informationen arbeiten oder es treten Situationen ein, die bei der Konzipierung der Systeme nicht berücksichtigt werden können. Dabei sinkt die Leistungsfähigkeit dieser Systeme.

Im Rahmen dieser Arbeit werden Robustheit und Generalisierbarkeit eines algorithmischen Ansatzes und eines Reinforcement Learning (RL) basierten Ansatzes zur Lösung des Brettspiels \glqq Vier Gewinnt\grqq{} untersucht. Bei Robustheit und Generalisierbarkeit handelt es sich um Eigenschaften, die beschreiben, wie gut ein Algorithmus oder RL-Modell in der Praxis funktioniert, in der andere Bedingungen herrschen können als während der Entwicklung und Qualitätssicherung. Diese Kriterien sind besonders relevant für den Erfolg von Algorithmen und Modellen in der Praxis.

Spiele eignen sich zur Untersuchung von Algorithmen und Modellen, weil sie reale Probleme auf kontrollierbare Umgebungen abstrahieren und gleichzeitig reproduzierbare und vergleichbare Messungen ermöglichen. Die Untersuchungen dieser Arbeit erfolgen am Beispiel des Brettspiels \glqq Vier Gewinnt\grqq{}, da aus früheren Untersuchungen ersichtlich wird, dass sich dafür sowohl algorithmische als auch Reinforcement Learning basierte Lösungen eignen.

Es wird Grundlagenforschung zu verbreiteten algorithmischen Ansätzen und Reinforcement Learning basierten Ansätzen betrieben. Anschließend werden die Aspekte Robustheit und Generalisierbarkeit von zwei Ansätzen aus den jeweiligen Bereichen am Beispiel von Vier Gewinnt empirisch untersucht. Dabei werden neue Erkenntnisse über Lösungsansätze von Vier Gewinnt herausgearbeitet, die sich auf vergleichbare Szenarien in der realen Welt übertragen lassen.

Die zentrale Fragestellung lautet: Inwiefern sind bei Vier Gewinnt algorithmische oder Reinforcement Learning basierte Ansätze robuster oder besser generalisierbar? Das Ziel dieser Arbeit besteht darin, ein detailliertes Verständnis über verschiedene Aspekte von Robustheit und Generalisierbarkeit der zu untersuchenden Ansätze zu bekommen.
	
	\section{Grundlagen}
    In diesem Kapitel werden Grundlagen zu … vermittelt, auf die im weiteren Verlauf dieser Arbeit Bezug genommen wird.
	\subsection{Automatisierung von Spielen}
	% Hier geht es um Algorithmen, die Bäume durchsuchen, also erstmal.

\subsubsection{Minimax}

Minimax (auch Minmax) ist ein Algorithmus, der ausgehend von einem Knoten im Spielbaum die darauf folgenden Knoten bewertet und den Knoten mit der besten Bewertung zurückgibt. Bei der Bewertung wird davon ausgegangen, dass der Gegner ebenfalls den Zug wählt, der für sich am günstigsten ist. Das führt dazu, dass wenn die Bewertung anhand der Gewinnchancen erfolgt, auch tatsächlich die Gewinnchancen maximiert werden.

Um die Gewinnchancen zu ermitteln, müssen jedoch alle Knoten des Spielbaums untersucht werden. Die Laufzeit des Algorithmus steigt linear zur Anzahl der zu untersuchenden Knoten und damit bei konstanter Anzahl von Möglichkeiten pro Zug exponentiell zur Suchtiefe. Den gesamten Spielbaum zu durchsuchen, ist daher nur für wenig komplexe Spiele praktikabel. Für komplexere Spiele muss die Suchtiefe begrenzt und auf Heuristiken zurückgegriffen werden, damit die Bewertung in akzeptabler Zeit erfolgen kann\cite{Ferguson.January2019}\cite{Heineman.October2008}.

\subsubsection{AlphaBeta}

Bei AlphaBeta handelt es sich um eine Erweiterung von Minimax.

	\subsection{Vier Gewinnt}
    Vier Gewinnt ist ein Brettspiel, das aus einem 7 x 6 Spielfeld besteht. Die beiden Spieler werfen abwechselnd einen Spielstein in eine Spalte hinein, der in dieser Spalte bis zur untersten freien Position fällt. Es gewinnt der Spieler, der als erstes vier Spielsteine in einer horizontalen, vertikalen oder diagonalen Reihe nebeneinander stehen hat\cite{MiltonBradleyCompany.1990}.

Bei Vier Gewinnt handelt es sich um ein kombinatorisches Nullsummenspiel für zwei Spieler. Kombinatorische Spiele weisen \glqq perfekte Information \grqq{} auf. Das bedeutet, dass alle Spieler zu jeder Zeit den gesamten Zustand des Spiels kennen. Bei kombinatorischen Spielen sind außerdem keine Zufallselemente enthalten. Die einzige Herausforderung beim Spielen kombinatorischer Spiele besteht darin, unter einer Vielzahl von Entscheidungsoptionen diejenige auszuwählen, die den besten weiteren Spielverlauf verspricht(\cite{Bewersdorff.2018}, S. 96-100)(\cite{Ferguson.January2019}, Kapitel 4.1).

Bei Zwei-Spieler-Nullsummenspielen, verursacht der Gewinn eines Spielers zwangsläufig einen Verlust des anderen Spielers. Die beiden Spieler haben also entgegengesetzte Interessen(\cite{Bewersdorff.2018}, S.100)(\cite{Allis.1994}, S. 100). Das bedeutet, dass sich der Erfolg von verschiedenen Lösungsansätzen durch die durchschnittliche Gewinnrate im Spiel gegeneinander bewerten lässt. Bei Nullsummenspielen mit mehr als zwei Personen, kann es passieren, dass wenn ein Spieler (bewusst oder versehentlich) nicht optimal spielt, ein zweiter Spieler davon profitiert, während ein dritter Spieler dadurch benachteiligt wird. Solche Wechselwirkungen sind bei Zwei-Personen-Nullsummenspielen ausgeschlossen(\cite{Bewersdorff.2018}, S. 113 ff.). Das macht die Messergebnisse im Hauptteil besser vergleichbar.

Nach Victor Allis lässt sich die Komplexität eines Spiels von strategie-basierten Zwei-Spieler-Nullsummenspielen durch ihre Zustandsraumkomplexität und Spielbaumkomplexität beschreiben. Die Zustandsraumkomplexität entspricht der Anzahl der verschiedenen möglichen Spielfeldkonfigurationen ab dem Start. Für ein Spiel kann dieser Wert oder zumindest dessen obere Schranke bestimmt werden, indem zunächst alle Konfigurationen des Spielfelds gezählt, dann Einschränkungen wie Regeln und Symmetrie berücksichtigt werden, und die Anzahl der illegalen und redundanten Zustände von der Anzahl aller möglichen Konfigurationen abgezogen wird.

% todo: Erklärung des Spielbaums

Die Spielbaumkomplexität beschreibt die Anzahl der Blattknoten des Lösungsbaums. Der Lösungsbaum beschreibt die Teilmenge des Spielbaums, der benötigt wird, um die Gewinnaussichten bei optimaler Spielweise beider Spieler zu berechnen. Die Spielbaumkomplexität lässt sich durch die durchschnittliche Spiellänge und der Anzahl der Entscheidungsmöglichkeiten pro Zug (entweder konstant oder abhängig vom Spielfortschritt) approximieren.

Die Spielbaumkomplexität ist maßgeblich für die praktische Berechenbarkeit einer starken Lösung. Für Tic Tac Toe wurde durch Allis eine obere Grenze für die Spielbaumkomplexität von 362880 ermittelt und eine starke Lösung lässt sich innerhalb von Sekundenbruchteilen berechnen (?). Für Schach wird die Spielbaumkomplexität auf $10^{31}$ geschätzt und die Aussichten auf eine starke Lösung liegen noch in weiter Ferne(\cite{Schaeffer.2007}).

% todo: Quelle für schnelle Lösbarkeit von vier Gewinnt

Für Vier Gewinnt wurde eine durchschnittliche Spiellänge von 36 Zügen und eine durchschnittliche Anzahl von Entscheidungsmöglichkeiten (freie Spalten) von 4 ermittelt. Damit wurde die Spielbaumkomplexität auf $4^{36} \approx 10^{21}$ geschätzt.

Verschiedene Lösungsverfahren von Vier Gewinnt sind bereits ausgiebig untersucht. Das Spiel wurde 1988 von James Dow Allen und Victor Allis unabhängig voneinander mit wissensbasierten Methoden schwach gelöst, was bedeutet, dass für die Anfangsposition eine optimale Strategie ermittelt wurde. Im Fall von Vier Gewinnt kann der Spieler, der den ersten Zug macht, bei optimaler Spielweise immer gewinnen.\cite{Allen.2010}\cite{Allis.1988}.

1993 wurde das Spiel durch Tromp auch durch einen Brute-Force Ansatz stark gelöst. Bei der Lösung durch John Trump kam Alpha-Beta-Pruning zum Einsatz, um für alle 4.531.985.219.092 legalen Zustände des Spiels den optimalen Zug zu berechnen\cite{Tromp}. Das hat 40.000 CPU Stunden gedauert. 

% todo: Bessere Quelle für Tromp

Lösungen, die alle Möglichkeiten durchrechnen, sind für den Einsatz in der Praxis aufgrund des hohen Rechenaufwands bei komplexeren Anwendungen auch heute noch selten praktikabel. Aus diesem Grund wird bevorzugt auf gute Heuristiken zurückgegriffen, die den Rechenaufwand minimieren, aber dennoch gute Ergebnisse liefern(\cite{Heineman.October2008}, Kapitel 7.6).

% todo: Erklärung grober Lösungsmöglichkeiten

Verschiedene allgemeine algorithmische und RL-basierte Ansätze wurden am Beispiel von Vier Gewinnt auf ihre Leistung untersucht. Es wurde gezeigt, dass sich sowohl algorithmische als auch RL-Ansätze bei Vier Gewinnt eignen\cite{Alderton.2019}\cite{Thill.2012}\cite{Wäldchen.2022}\cite{Taylor.2024}\cite{Sheoran.2022}\cite{Qiu.2022}.


    \section{Konzept}
    In diesem Kapitel wird erklärt, wie eine Messumgebung aufgesetzt wurde, um die Robustheit der Lösungsverfahren empirisch zu bewerten. Diese Messumgebung ermöglicht es, zwei Agenten, die die zu untersuchenden Ansätze implementieren, das Spiel wiederholt gegeneinander spielen zu lassen. Die Spiele werden unter verschiedenen Szenarien durchgeführt, die jeweils auf ein in Kapitel \ref{robustheit} definierten Aspekt der Robustheit abzielen. Es werden die Gewinnraten gemessen, worüber Aussagen darüber getroffen werden können, welches der beiden Verfahren in den verschiedenen Szenarien stärker und damit robuster ist. Folgende Szenarien werden untersucht:

\begin{itemize}
	\item Unsicherheit bezüglich Aktionen: Die Aktionen der Agenten bestehen darin, dass sie den nächsten Spielstein in eine freie Spalte des Spielfelds hineinwerfen, die sie auf Grundlage des beobachteten Spielfeldzustands auswählen. Dieses Szenario führt Unsicherheit bezüglich Aktionen ein, indem für einen Agenten der Spielstein mit einer bestimmten Wahrscheinlichkeit nicht in die ausgewählte Spalte, sondern in eine zufällige freie Spalte fällt.
	
	\item Unsicherheit bezüglich Beobachtungen: Die Agenten wählen zwischen möglichen Aktionen basierend auf deren Beobachtungen des Spielfelds. In diesem Szenario erhalten die Agenten fehlerhafte Informationen über das Spielfeld. Jedes Feld besitzt dabei eine bestimmte Wahrscheinlichkeit mit der nicht dessen tatsächlicher Zustand (leer, besetzt durch Spieler 1, besetzt durch Spieler 2) erkannt wird, sondern ein zufälliger Zustand.
	
	Es ist anzumerken, dass hierbei aus dem MDP ein Partially Observable MDP (POMDP) wird, also ein MDP dessen Umgebung nur teilweise oder fehlerhaft beobachtbar ist. Der betroffene Agent kann nicht mit Sicherheit gesagt werden, in welchem Zustand er sich gerade befindet. Zusätzlich zum MDP enthält das POMDP ein Observation Modell O(s, o), das die Wahrscheinlichkeit beschreibt, eine Beobachtung o im Zustand s zu machen. POMDPs sind wesentlich komplizierter gezielt zu lösen, in der realen Welt jedoch wesentlich häufiger anzutreffen. Es gibt Optimierungen von MCTS und bestimmte RL-Methoden, über die POMDPs gezielt gelöst werden können, zum Beispiel indem für eine Entscheidung nicht nur der aktuelle Zustand, sondern die Historie der Zustände betrachtet wird (\cite{Russell.2020}, S. 588 ff.). Da es in dieser Arbeit darum geht, zu untersuchen, inwiefern grundsätzliche Eigenschaften von symbolischen Algorithmen und Reinforcement Learning Robustheit beeinflussen, werden diese gezielten Lösungen zur Vereinfachung nicht betrachtet. Beide Agenten gehen davon aus, dass es sich bei dem fehlerhaften Bild um den tatsächlichen Zustand des Spiels handelt, auch wenn der beobachtete Zustand gemäß der Spielregeln nicht erreicht werden könnte.
	
	\item Unsicherheit bezüglich Dynamik der Umgebung: Die Agenten erwarten ein bestimmtes Verhalten von der Umgebung, das durch die Spielregeln abgebildet wird. Der MCTS-Agent führt Simulationen anhand dieser Erwartungen aus und der RL-Agent wird auf Grundlage dieser Erwartungen trainiert. In diesem Szenario werden die Erwartungen an das Verhalten der Umgebung gebrochen, indem ein bestimmter Spieler mit einer bestimmten Wahrscheinlichkeit zwei Züge hintereinander durchführt.
	
	% \item Unsicherheit bezüglich Dynamik der Umgebung: Mit einer bestimmten Wahrscheinlichkeit führt der Gegenspieler nicht den Zug aus, den er für am besten hält, sondern einen zufälligen Zug. (Wurde rausgenommen, weil das quasi das gleiche ist wie Unsicherheit bezüglich Aktionen) nur von der anderen Perspektive.
\end{itemize}

In den verschiedenen Szenarien gelten die veränderten Bedingungen nur für jeweils einen Agenten. Wenn beide Agenten gleichzeitig von den veränderten Bedingungen betroffen wären, könnten nur Aussagen über die relative Robustheit zueinander getroffen werden. Dadurch dass die veränderten Bedingungen nur für jeweils ein Verfahren auf einmal gelten, kann genau gesagt werden, welches Verfahren wie stark betroffen ist.

Die Messungen werden nicht nur unter Szenarien mit veränderten Bedingungen durchgeführt, sondern auch in einer neutralen Umgebung ohne veränderte Bedingungen, um eine Grundlage für die Analyse der Ergebnisse zu bilden.

Um einen fairen Vergleich zu gewährleisten, müssen die Agenten in der neutralen Umgebung gleich stark sein. Dazu werden die Parameter der Agenten (Anzahl der Simulationen beim MCTS Agenten, Anzahl der durchlaufenen Self-Play-Trainingsepisonden beim RL-Agenten) so eingestellt, dass sie in der neutralen Umgebung im Spiel gegeneinander jeweils eine Gewinnrate von etwa 50\% aufweisen.

Außerdem ist es wichtig, dass die Agenten auf einem gewissen starken Level spielen. Denn bei Agenten, die ohnehin nicht stark spielen, wird es schwierig sein, in den verschiedenen Szenarien zur Untersuchung der Robustheit eine aussagekräftige Änderung in den Gewinnraten zu messen. Als Indikator für die Spielstärke der Agenten wird die durchschnittliche Spieldauer verwendet. Es wird angenommen, dass starke Agenten weniger Fehler machen und ihre Züge strategischer wählen, sodass komplexere Spielsituationen entstehen und sich die Entscheidung des Spiels hinauszögert.

Da bei Vier Gewinnt, wie in Kapitel \ref{vier-gewinnt} erwähnt, der Spieler, der den ersten Stein platziert, einen Vorteil hat, wechselt zu Beginn jedes Spiels das Recht, den ersten Zug zu machen.

    
    \section{Realisierung}
    % PettingZoo-Bibliothek

Als Grundlage für die Messumgebung dient die Open Source Python-Bibliothek PettingZoo, die zum Entwickeln und Testen von MARL-Systemen konzipiert wurde. Sie stellt eine einheitliche Schnittstelle zu Umgebungen bereit, in der Agenten miteinander interagieren können.

Die Umgebungen definieren den Rahmen, in dem die Agenten miteinander interagieren. Sie weisen ein bestimmtes Verhalten auf und definieren unter anderem, unter welchen Bedingungen welche Aktionen möglich sind, Belohnungen verteilt werden und in welcher Form die Agenten Informationen über den Zustand der Umgebung erhalten. Es existieren eine Reihe von vorgefertigten Umgebungen, darunter welche, die kooperative Probleme zum Benchmarking von MARL-Systemen oder auch rundenbasierte Spiele wie Vier Gewinnt abbilden.

Die durch PettingZoo bereitgestellte Schnittstelle ermöglicht es, aus Sicht eines Agenten den aktuellen Zustand der Umgebung zu beobachten, die im aktuellen Zustand möglichen Aktionen und erhaltenen Belohnungen zu ermitteln und eine Aktion auszuwählen, die in der Umgebung durchgeführt werden soll.

Für alle Agenten der Umgebung kann benutzerdefinierte Logik eingebunden werden, die bestimmt, wie sie ihre Aktionen wählen. Vorgesehen sind dabei RL-Modelle, es können jedoch auch symbolische Algorithmen eingesetzt werden, darunter auch welche, die ihre Entscheidungen rein zufällig oder unter Einbezug von menschlichen Eingaben treffen \cite{Farama.2025}.

% Implementierung der Messumgebung

Die Implementierung der Messumgebung baut auf der offiziellen Implementierung der Vier-Gewinnt-Umgebung von PettingZoo auf. Die Messumgebung tut dabei nichts anderes, als wiederholt zwei Agenten mit bestimmten Lösungsansätzen das Spiel spielen zu lassen und dabei die Gewinnraten und Spieldauer zu messen. Die Open-Source-Eigenschaft ermöglicht es, den Quellcode zu modifizieren. Davon wird im weiteren Verlauf der Arbeit Gebrauch gemacht, unter anderem, um die verschiedenen Szenarien zur Untersuchung von Robustheit abzubilden.

Im Zuge der Realisierung der Messumgebung ist aufgefallen, dass wenn zwei Agenten (Spieler 0 und Spieler 1) alle Aktionen im Spiel mit derselben Wahrscheinlichkeit rein zufällig wählen, nach 1000 Spielen Spieler 0 mit 55,20 \% gegenüber Spieler 1 mit 44,30 \% eine wesentlich höhere Gewinnrate erzielt.

\begin{figure}[ht!]%[!tbp]
	\begin{subfigure}[b]{0.48\textwidth}
		\includegraphics[width=\textwidth]{Bilder/constant_player_order_graph_win_rates.png}
		\caption{Gewinnrate.}
		\label{fig:f1}
	\end{subfigure}
	\hfill
	\begin{subfigure}[b]{0.48\textwidth}
		\includegraphics[width=\textwidth]{Bilder/constant_player_order_graph_game_length.png}
		\caption{Durchschnittliche Spieldauer.}
		\label{fig:f2}
	\end{subfigure}
	\caption{Gewinnrate und durchschnittliche Spieldauer bei konstanter Spielerreihenfolge.}
\end{figure}

Das lässt sich dadurch erklären, dass die Vier-Gewinnt-Umgebung so implementiert ist, dass Spieler 0 immer der Spieler ist, der den ersten Stein setzen darf. Er ist damit seinem Gegenspieler immer einen Spielzug voraus, was die Wahrscheinlichkeit erhöht, als erstes Vier Steine in eine Reihe zu bekommen. An dieser Stelle sei nochmals zu erwähnen, dass bei Vier Gewinnt der erste Spieler bei optimaler Spielweise stets gewinnen kann. Um ausgeglichene Messungen zu gewährleisten, muss daher sichergestellt werden, dass sich im Rahmen der Messungen die beiden Spieler mit dem ersten Zug abwechseln.

Die Vier-Gewinnt-Umgebung von PettingZoo wurde daher erweitert, um einen Parameter entgegenzunehmen und zu verarbeiten, der bestimmt, welcher Spieler anfangen soll. Die Messumgebung wechselt den Wert des Parameters nach jedem Spiel durch. Nach dieser Änderung weisen die Spiele wesentlich ausgeglichenere Ergebnisse auf. Spieler 0 gewinnt 50,30 \% und Spieler 1 49,50 \% der Spiele. Die durchschnittliche Spieldauer bleibt dabei mit 22,48 Zügen vor der Änderung gegenüber 22,26 Zügen nach der Änderung nahezu unverändert.

\begin{figure}[ht!]%[!tbp]
	\begin{subfigure}[b]{0.48\textwidth}
		\includegraphics[width=\textwidth]{Bilder/alternating_player_order_graph_win_rates.png}
		\caption{Gewinnrate.}
		\label{fig:f3}
	\end{subfigure}
	\hfill
	\begin{subfigure}[b]{0.48\textwidth}
		\includegraphics[width=\textwidth]{Bilder/alternating_player_order_graph_game_length.png}
		\caption{Durchschnittliche Spieldauer.}
		\label{fig:f4}
	\end{subfigure}
	\caption{Gewinnrate und durchschnittliche Spieldauer bei abwechselnder Spielerreihenfolge.}
\end{figure}

% Es kommen Reinforcement Learning Modelle zum Einsatz, die aus RL-Bibliotheken wie CleanRL oder Stable-Baselines bereitgestellt werden. Falls vorhanden, wird auf fertig implementierte Algorithmen zurückgegriffen.

\newpage
    
    \section{Ergebnisdiskussion}
    Um die Robustheit der im Rahmen dieser Arbeit implementierten MCTS- und PPO-Agenten miteinander zu vergleichen, wurden sie in den in Kapitel \ref{robustheit-szenarien} beschriebenen Szenarien im Spiel gegen einen zufällig spielenden Agenten evaluiert. Die Agenten spielen dabei mit abwechselndem Anzugsrecht gegen einen zufällig spielenden Agenten. Gemessen wird die Gewinnrate der zu untersuchenden Agenten in Abhängigkeit des Ausmaßes der in den Szenarien eingeführten Unsicherheit.

Für jeden untersuchten Unsicherheitsgrad wurden mit dem PPO-Agenten 2000 Spiele durchgeführt. Bei den Untersuchungen mit dem MCTS-Agenten wurden auf Grund des erhöhten Rechenbedarfs und der begrenzten Zeit nur 200 Spiele durchgeführt. Wie in Kapitel \ref{konzept} beschrieben, wird der Gewinnratenverlust als relatives Maß für die Robustheit der Verfahren verwendet. Ein höherer Gewinnratenverlust deutet dabei auf ein weniger robustes Verfahren hin. Die tatsächlich gemessenen Gewinnraten im Spiel gegen den zufällig spielenden Agenten befinden sich im Anhang unter ref{}.

\subsection{Unsicherheit bezüglich Aktionen}

Beim Vergleich der Verfahren hinsichtlich der Robustheit gegenüber Unsicherheiten bezüglich Aktionen wurde die Wahrscheinlichkeit, anstelle des durch den Agenten gewählten Zuges einen zufälligen Zug durchzuführen, in 10-Prozentpunktschritten von 0 \% bis 100 \% variiert.

\begin{figure}[ht!]%[!tbp]
	\includegraphics[width=0.7\textwidth, center]{Bilder/robustness-results/uncertain_actions_win_rate_losses.png}
	\caption{Gewinnratenverlust in Abhängigkeit von der Wahrscheinlichkeit, eine zufällige Aktion durchzuführen.}
\end{figure}

Beträgt diese Wahrscheinlichkeit 0 \%, befindet sich der Gewinnratenverlust bei beiden Agenten ebenfalls bei 0 \%, da es sich bei den dabei erzielten Gewinnraten um den Ausgangswert für die Berechnung des Gewinnratenverlustes handelt.

Ab einer Wahrscheinlichkeit für zufällige Aktionen von 20 \% ist für den MCTS-Agenten ein signifikant niedrigerer Verlust von … gegenüber von … bei PPO zu verzeichnen. Einen solche signifikante Differenz hält sich bis zu einer Wahrscheinlichkeit von 80 \% für zufällige Aktionen, wo ein Gewinnratenverlust von … beim MCTS-Agenten und … beim PPO-Agenten ermittelt wurde. Damit ist der MCTS-Agent in diesem Bereich signifikant robuster als der PPO-Agent.

Bei 90 \% liegen die Messpunkte von MCTS zwar mit … gegenüber … zwar weiterhin höher, aber die Konfidenzintervalle der Messpunkte sind so nah bei einander, dass aus den Werten allein nicht mit ausreichender Sicherheit Schlüsse über die tatsächlichen Werte gezogen werden können.

Bei einer Wahrscheinlichkeit für zufällige Aktionen von 100 \% sollten beide Agenten nach genügend durchgeführten Spielen gegen den zufällig spielenden Agenten eine Gewinnrate von 50 \% und damit einen Gewinnratenverlust von 100 \% erzielen. Die Messpunkte liegen für MCTS mit … leicht darüber bzw. für PPO mit … leicht darunter. Da ein Gewinnratenverlust von 100 \% innerhalb der Konfidenzintervalle liegt, ist es naheliegend, dass die gemessenen Abweichungen stochastisch bedingt sind.

Aus den Messungen geht für eine Wahrscheinlichkeit für zufällige Aktionen von 20 \% bis 80 \% der MCTS-Agent als der robustere Agent hervor. Für darunter bzw. darüber liegende Wahrscheinlichkeiten kann aufgrund von Messungenauigkeiten kein signifikanter Unterschied festgestellt werden. Der Verlauf des Gewinnratenverlusts liegt jedoch nahe, dass MCTS auch dort weiterhin einen niedrigen tatsächlichen Verlust aufweist.

Es wurde nicht erwartet, im Szenario mit Unsicherheiten bezüglich Aktionen eine signifikante Differenz im Gewinnratenverlust zwischen den beiden Verfahren zu sehen, die durch die Verfahren selbst bedingt sind. Es liegt nahe, dass die Diskrepanz dadurch hervorgerufen wird, dass der PPO-Agent unzureichend trainiert wurde, wodurch er eine kurzfristigere Strategie besitzt, die anfälliger für Störungen ist. Daher können aus den Messungergebnissen keine Schlüsse darüber gezogen werden, welches der beiden Verfahren grundsätzlich robuster gegenüber Unsicherheit bezüglich Aktionen ist. Somit sind in dieser Hinsicht auch keine verallgemeinernden Aussagen über symbolische Algorithmen und Reinforement Leraning möglich.

\subsection{Unsicherheit bezüglich Beobachtungen}

Beim Vergleich der Verfahren hinsichtlich der Robustheit gegenüber Unsicherheiten bezüglich Beobachtungen wurde die Anzahl von fehlerhaften Spielsteinplatzierungen, in 2er-Schritten von 0 bis 20 variiert. Aufgrund der größer werdenden Konfidenzintervalle wurden ab 10 fehlefhaften Spielsteinplatzierungen nicht mit 200 sondern 500 Spiele zwischen dem MCTS-Agenten und dem zufällig spielenden Agenten ausgetragen.

\begin{figure}[ht!]%[!tbp]
	\includegraphics[width=0.7\textwidth, center]{Bilder/robustness-results/uncertain_observations_win_rate_losses.png}
	\caption{Gewinnratenverlust in Abhängigkeit von der Anzahl fehlerhafter Spielfeldplatzierungen.}
\end{figure}

Ähnlich wie in … startet der Gewinnratenverlust beider Agenten bei 0 fehlerhaften Spielsteinplatzierungen mit 0 \%.

Ab 4 fehlerhaften Spielsteinplatzierungen lässt sich bei MCTS mit … ein signifikant niedrigerer Gewinnratenverlust gegenüber PPO mit … verzeichnen. Diese Diskrepanz hält sich bis zu einer Anzahl von 10 fehlerhaften Platzierungen, wobei der MCTS-Agent einen Gewinnratenverlust von … und der PPO-Agent einen Verlust von … erzielt. Das zeigt, dass sich der MCTS-Agent in diesem Bereich signifikant robuster verhält als der PPO-Agent.

Ab 12 fehlerhaften Platzierungen sind die Differenzen zwischen den Messpunkten zu niedrig und die Konfidenzintervalle so groß, dass nicht mit ausreichender Sicherheit bestimmt werden kann, bei welchem Agenten die tatsächlichen Werte höher oder niedriger liegen.

Bei 18 und 20 fehlerhaft platzierten Steine liegt der gemessene Verlust von MCTS mit … und sogar über dem von PPO mit … und …, aufgrund der großen Konfidenzintervalle kann jedoch nicht gesagt werden, ob sich die tatsächlichen Werte genauso verhalten. Um Aussagen darüber treffen zu können, sind Messungen mit mehr Wiederholungen nötig.

Bei 4 bis 10 fehlerhaften Spielsteinplatzierungen geht der MCTS-Agent als der robustere Agent hervor. Aufgrund des Verlaufs des Gewinnratenverlusts in Abhängigkeit er fehlerhaften Spielsteinplatzierungen liegt nahe, dass dies auch für unter einer Anzahl von 4 gilt. Liegt die Anzahl der fehlerhaften Spielsteinpkatzierungen über 10 können aufgrund der zu großen Messungenauigkeiten keine signifikanten Unterschiede festgestellt werden. Die beobachtete höhere Robustheit des MCTS-Agenten könnte durch die Anfälligkeit von neuronalen Netzwerken gegenüber unerwarteten Beobachtungen verursacht worden sein. Ebenso wie in … könnte die Ursache für dieses Verhalten jedoch auch in der kurzfristigen Strategie des in dieser Arbeit unzureichend trainierten PPO-Modells liegen. Damit können auch hier keine Aussagen darüber getroffen werden, welches der beiden Verfahren MCTS oder PPO bzw. symbolischer Algorithmus oder Reinforcement Learning grundsätzlich robuster gegenüber Unsicherheiten bezüglich Beobachtungen ist.

Auch wenn aufgrund des unzureichenden Trainings des in der Arbeit implementierten PPO-Agenten keine Aussagen getroffen werden können, die auf die Verfahren verallgemeinert werden können, ist anzumerken, keine Überraschung in der Hinsicht aufgetreten ist, als dass der PPO-Agent trotz seiner unterlegenen Strategie als signifikant robuster gemessen wurde als der MCTS-Agent.

Um Aussagen über das Verhalten der Agenten auf die Verfahren MCTS und PPO bzw. symbolischer Algorithmus und Reinforcement Learning verallgemeinern zu können, muss sichergestellt sein, dass die Agenten unter neutralen Bedingungen möglichst die selbe Leistung erzielen. Im Rahmen dieser Arbeit konnten durch quantitative und qualitative Analysen der Agenten gezeigt werden, dass dies in diesem Fall nicht zutrifft. Es ist kritisch zu hinterfragen, inwiefern diese Methodiken ausreichen würden, um zu zeigen, dass die Agenten dieselbe Leistung erzielen. Denn im Spiel gegen einen zufällig spielenden Agenten spiegeln sich große Leistungsunterschiede nur in sehr kleinen Differenzen wieder. Und in der qualitativen Analyse können subjektive Verzerrungen der Wahrnehmung das Ergebnis beeinflussen. Besser geeignet sind beispielsweise das Spiel gegen einen perfekt spielenden Agenten wie in … oder gegen unterschiedliche nicht-perfekt spielende Agenten in einer Turnier-Umgebung, so wie es in … umgesetzt wurde.




    
    \section{Zusammenfassung und Ausblick}
    Aus den im Rahmen dieser Arbeit durchgeführten Experimenten geht hervor, dass der in dieser Arbeit implementierte MCTS-Agent robuster hinsichtlich Unsicherheiten gegenüber Aktionen ist als der in dieser Arbeit implementierte PPO-Agent. Hinsichtlich Robustheit bezüglich Unsicherheiten gegenüber Beobachtungen konnte dies aufgrund von stochastisch bedingten Messungenauigkeiten bis zu einem gewissen Grad an Unsicherheit ebenfalls festgestellt werden. Aufgrund begrenzter Ressourcen war es nicht möglich, einen PPO-Agenten zu trainieren, der in der neutralen Umgebung ohne Unsicherheiten mit dem MCTS-Agenten mithalten kann. Es ist davon auszugehen, dass die kurzfristigere Strategie des PPO-Agenten anfälliger für die in den Szenarien zur Untersuchung von Robustheit eingeführten Unsicherheiten ist. Die Beobachtung, dass der MCTS-Agent robuster als der PPO-Agent ist, kann damit nicht auf die Verfahren MCTS und PPO und schon gar nicht auf symbolische Algorithmen und Reinforcement Learning verallgemeinert werden. Damit wurde nur ein Teilbereich der zu untersuchenden Frage beantwortet, inwiefern symbolische Algorithmen oder RL-Verfahren robuster sind.

Mit der in dieser Arbeit angewandten Methodik war es jedoch möglich, die Robustheit von zwei konkreten Implementierungen zu quantifizieren. Es ist naheliegend, dass auch weitere Teilbereiche der zu untersuchenden Frage beantwortet werden können, indem die Methodik auf weitere MDPs und entsprechenden Lösungsverfahren übertragen wird. Diese erfordert lediglich, dass das zu untersuchende MDP gezielt auf Aspekte von Robustheit modifiziert, die zu untersuchenden Lösungen auf das modifizierte MDP angewandt und der Verlust ihrer Leistungsfähigkeit untersucht wird.

Um die Frage zu beantworten, wie sich der Vergleich der Robustheit von MCTS und PPO bei Vier Gewinnt allgemein verhält, ist es notwendig, zwei Agenten zu untersuchen, die unter neutralen Bedingungen möglichst gleich stark sind. Für das Training eines stärkeren PPO-Agenten scheint die Trainingsstrategie von Zhong et al. in Kombination mit mit Faltungsmatritzen ausgestatteten neuronalen Netzwerken einen vielversprechenden Ansatz darzustellen. Um Aussagen über die Robustheit von MCTS und PPO über verschiedene Anwendungsfälle verallgemeinern zu können, sind Untersuchungen mit anderen Anwendungsfällen notwendig, die beispielsweise verschiedene Zustands- und Aktionsräume aufweisen oder die Möglichkeit bieten, physikalische Parameter der Problemumgebung zu variieren. Aussagen über den Vergleich von Robustheit zwischen symbolische Algorithmen und RL-Verfahren allgemein erfordern Untersuchungen mit ausreichend verschiedenen Verfahren und Anwendungsszenarien. Hierbei stellt sich jedoch die Frage, inwiefern dies aufgrund der Vielzahl von zu lösenden Problemen und entsprechenden Lösungsverfahren möglich oder überhaupt erstrebenswert ist.

 
	\section{Literaturverzeichnis}
	\nocite{*}
    \renewcommand{\refname}{}
	\printbibliography
 
	\newpage
	
\end{document}
